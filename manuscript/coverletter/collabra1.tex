
\documentclass{stanfordletter}
%\makelabels
\usepackage{url}
\begin{document}
	\name{Veronica Boyce}
	\signature{\vspace{-35pt} Veronica Boyce, \\ on behalf of the author team}
	
	
	\begin{letter}{Professor Don van Ravenzwaaij \\ Editor-in-Chief \\ Collabra: Psychology }
		
		
          \opening{Dear Dr. van Ravenzwaaij,} 
          This submission contains the manuscript ``Estimating the replicability of psychology experiments after an initial failure to replicate'' which we are submitting to Collabra: Psychology for publication as an original research report.
          
          The manuscript reports on 17 re-replications of psychology experiments for which an original replication had failed.  The original failures to replicate come from Boyce et al. 2023 (\url{https://doi.org/10.1098/rsos.231240}), and the current work is a follow-up to that study.  
          
          As a practical matter, scientists in disciplines with low replicability rates (like psychology) will often face the question of what to do after a failed replication attempt. When does it make sense to try again and when does it make sense to move on? There is a very limited literature on how successful re-replications that attempt to ``rescue'' a failed replication are -- we are only aware of Many Labs 5 (Ebersole et al. 2020) which re-replicated 11 experiments from the Reproducibility Project: Psychology (Open Science Consortium, 2015) in high-resource settings with expert input. 
          
          We build on this by looking at the rate of re-replication when both the first replication and re-replication are from early career researchers. Our current study provides an estimate of how often one can successfully ``rescue'' an effect after a failed replication by addressing possible causes of the first replication's failure. In 5/17 of these ``rescue'' projects (29\%), the ``rescue'' study mostly or fully replicated the original results, albeit with a smaller effect size; in the other 12, the second replication also failed.
          
          We speculate that successful rescue projects were due to larger sample sizes or methodological changes such as attention checks. and we discuss some of the individual projects as case studies. 
           
          This work has been uploaded to the preprint server PsyArxiv at \url{https://osf.io/preprints/psyarxiv/an3yb}. 
          
          
          This manuscript is not under review at any other venue. Please let us know if you need any further information in
          connection with this submission. 
          
          We look forward to hearing from you regarding the manuscript!
          
          \closing{Sincerely,}
		
%		Please provide a cover letter either as a file or pasted into the box when indicated. Cover letters should be no more than 1-page of A4 in length. Cover letters are only made available to the Editors.
%		
%		A good cover letter can help the Editor understand how the study relates to previously published work, and your motivations for completing the study and submitting it to Royal Society Open Science.
%		
%		Your cover should ideally include:
%		
%		An explanation of how your work is appropriate for Royal Society Open Science;
%		A summary of the scientific validity of the work. We do not make judgements about impact or interest in the journal, but we do expect papers to be high-quality, and to represent a meaningful advance in the field;
%		The reason for choosing the selected article type;
%		Any previous interactions you have had with Royal Society Open Science editorial office staff or Editors regarding the submitted manuscript (for instance, at a conference);
%		We welcome the use of preprint servers, but we encourage you to declare it if you have uploaded the manuscript to a preprint server prior to submitting to the journal;
%		Where applicable, a short explanation of the sampling techniques used, statistical tests employed, and sample sizes chosen (ie the statistical power of the study);
%		Any other matters you would like to make the Editors aware of.
%		
%		If similar papers to yours have been recently published in Royal Society Open Science, you may wish to include a link to one or more of these publications in your cover letter as additional context for choosing the journal to publish your study.
%		
%		Your cover letter should not be a repetition of your paper’s abstract or introduction. Your manuscript may be returned to you if you duplicate your abstract or introduction in your cover letter.
%		
%		As you will be asked to include preferred or non-preferred referee suggestions elsewhere in the submission process, you do not need to include these details in your cover letter, unless there are specific matters you wish to highlight to the Editors regarding your suggestions.
%		
%		Finally, prior to submitting your manuscript and cover letter, it is recommended that you not only check your spelling but that you have included the correct journal title within the cover letter. 
	\end{letter}
	
\end{document}




